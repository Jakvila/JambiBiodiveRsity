\documentclass{article}

\usepackage{Sweave}
\begin{document}
\input{panel_data_script-concordance}

\section{About p-values}
Unsignificant values don't permit us to draw the conclusion that there is no real effect.
Looking at papers in detail is like looking at the backyard of a slaughterhouse.
Only the p-value is almost never sufficient to draw meaningful conclusions on significance of a coefficient. At least we should take the effect size into account.

\subsection{Panel data}

Repeated observations of some individual unit over time.
standard case: The same individual over the same unit of time. -> balanced panel

But "atrition" often leads to unbalanced panels. 
Notation: Notation for subscript

i : index for individual observations

t : index for time periods



\[ X_i,t , ... \]

if N individuals for T time periods => sample size
NT: balanced panel

everything else: unbalanced people

Further examples:

Rotationg panels (Socio-Economic panel SOEP)

Pseudo panels ( mean cohort values over time): often used for poverty research in developing countries. Advantage: can combine data, once the cohort is identified. Deaton (1985)

Why panel data?
\begin{itemize}
\item{ more observations => more information}
\item{dynamic analysis: \item{shocks over time average out}}
\item{ \textbf{unobserved heterogeneity}}
\end{itemize}
\section{Notation:}

Linear Regression 

cross section \[ Y_i = \beta_0 + \beta_1*X_1i + u_i ... \]

panel structure \[ Y_{it} = \beta_0 + \beta_1*X_{1it} + u{it} ... \]

                \[ Y_{it} = \beta_{0t} +  \beta_1*X_{1it} + u{it} ... \]
                
                t = 1 \[ Y_{i1} = \beta_{01} + \beta_1*X_{1,1} + u_{i1} ... \]
                
                t = T \[ Y_{iT} = \beta_{0T} + \beta_{1T}*X_{1iT} + u_{iT} ... \]
                
                
                
                \[ D_{i} = 1 if i = j,  D_{i} = 0 if i != j for j = 1,..., N \]
                
                least-squares dummy variable estimator LSDV 
                
                
                
                
                with individual means over time.
                
                
               \[ (Y_{it} - \bar Y_{i0}) = \beta_1(X_{it}-x_{1i.}) + (u_{it} - \bar u_i) \]
               
               how to get \[ \hat \beta_{0i} \] ?
               \[ \hat \beta_{0i} = \bar Y_{i0} - \beta \]
                
                
                
\begin{Schunk}
\begin{Sinput}
> library(plm)
> setwd("C:/Users/jakob/OneDrive/University/Data_analysis_Oct19/Panel_data")
> dataNL <- readRDS("dataNL.rds")
> names(dataNL) <-   c("index", "year", "milk", "other", "x1", "x2", "x3", "x4", "x5", "trend")
> summary(dataNL)          
\end{Sinput}
\begin{Soutput}
     index          year           milk             other        
 Min.   :  1   Min.   :1.00   Min.   :-6.9852   Min.   :-2.6771  
 1st Qu.: 36   1st Qu.:1.75   1st Qu.:-1.6975   1st Qu.:-1.3216  
 Median : 71   Median :2.50   Median :-1.1217   Median :-0.9165  
 Mean   : 71   Mean   :2.50   Mean   :-1.4386   Mean   :-1.0416  
 3rd Qu.:106   3rd Qu.:3.25   3rd Qu.:-0.7616   3rd Qu.:-0.6472  
 Max.   :141   Max.   :4.00   Max.   :-0.3433   Max.   :-0.3227  
       x1              x2               x3               x4        
 Min.   :1.000   Min.   :0.1514   Min.   :0.3476   Min.   :0.1485  
 1st Qu.:1.018   1st Qu.:0.5597   1st Qu.:0.7416   1st Qu.:0.5993  
 Median :1.064   Median :0.9163   Median :0.9618   Median :0.9368  
 Mean   :1.126   Mean   :1.0000   Mean   :1.0000   Mean   :1.0000  
 3rd Qu.:1.186   3rd Qu.:1.2430   3rd Qu.:1.1890   3rd Qu.:1.3456  
 Max.   :1.896   Max.   :3.1780   Max.   :2.5610   Max.   :2.3695  
       x5             trend     
 Min.   :0.1643   Min.   :1.00  
 1st Qu.:0.6148   1st Qu.:1.75  
 Median :0.9223   Median :2.50  
 Mean   :1.0000   Mean   :2.50  
 3rd Qu.:1.3143   3rd Qu.:3.25  
 Max.   :3.3186   Max.   :4.00  
\end{Soutput}
\begin{Sinput}
> dataNL$lmilk <- -log(-dataNL$milk)
> dataNL$lx1 <- log(dataNL$x1)
> dataNL$lx2 <- log(dataNL$x2)
> dataNL$lx3 <- log(dataNL$x3)
> dataNL$lx4 <- log(dataNL$x4)
> dataNL$lx5 <- log(dataNL$x5)
>             
\end{Sinput}
\end{Schunk}
            The trend variable remains unlogged.
            
                <<>>=
            plot(dataNL$lmilk~dataNL$lx1)
            plot(dataNL$lmilk~dataNL$lx2)
            plot(dataNL$lmilk~dataNL$lx3)
            plot(dataNL$lmilk~dataNL$lx4)
            plot(dataNL$lmilk~dataNL$lx5)

            formula.NL <- lmilk ~ lx1 + lx2 + lx3 + lx4 + lx5 + trend
            
            lm.NL <- lm(formula.NL , data=dataNL)
                
                summary(lm.NL)
                
            Pool.NL <- plm(formula.NL, data = dataNL, model = "pooling")
            
            summary(Pool.NL)
            
            formula.LSDV <- lmilk ~ lx1 + lx2 + lx3 + lx4 + lx5 + trend + as.factor(index) # if we run that , index has ~ 140 dummy variables we run into the problem of perfect multicollinearity. So R automatically drops one of the dummies.
            
            lm.LSDV <- lm(formula.LSDV, data = dataNL)
      
            summary((lm.LSDV))
         
              @
            In order to extract a coefficient, we use the coef() function
            <<>>=
            
              coef(Pool.NL)[2:6]
              sum(coef(Pool.NL)[2:6])
              
              @
                output at 1.06 which is too high. Maybe we get different results with the LSDV estimator.
            <<>>=
            
              coef(lm.LSDV)[2:6]
              sum(coef(lm.LSDV)[2:6])
              
              @
              now lower coefficient taking the index dummies into account.
              
              <<>>=
            
             require(car)
              
              linearHypothesis(Pool.NL , "lx1+lx2+lx3+lx4+lx5=1")
              
              summary(Pool.NL)
              summary(lm(formula.NL, data = dataNL))
              sum(coef(Pool.NL)[2:6])
              
              WI.NL <- plm(formula.NL, data = dataNL, model = "within")
              cbind(coef(lm.LSDV[2:7], coef(WI.NL)))
              
              @
                About manually applying F-Tests :
                 - unrestricted (ignoring H0) - RSS^UR [residual sum of squares]
                 restricted( imposing H0) - RSS^R
                 
                 \[ \star F = \frac {RSS^R - RSS^UR / +1}{RSS^UR / (NT - (k-1))} \]
                 
                 
              Substract means from every variable.. Using loops (?)
              
              Dummy variables you cannot meaningfully de-mean over time. So we use the LM, but should get out the same                results as with the LSDV model.
              <<>>=
              # wi2.NL <- plm(formula.NL, data = dataNL, effect = "twoways", model = "within )
              
              
              plot(density(fixef(WI.NL)))
              
              @
          Problem: time-invariant variables and how to deal with them..
              <<>>=
              dataNL$TimeInvar <- runif(141) %x% rep(1, 4)
              formula.TimeInvar <- lmilk + lx1 + lx2 + lx3 + lx4 + lx5 + trend + TimeInvar
              
              
              head(dataNL$TimeInvar)  
                WI.NL <- plm(formula.NL , data = dataNL, model = "within")
                
                @
                Next steps: random effects model
                
                \section{scenario}
                
                No interest in the unobserved heterogeneity, no need to interpret the individual effects;
                 
                  \[ \alpha_i \] - parameters are a mere cuisance (guidance?) --> error
                  
                  \[ Y_{i,t} = \alpha_i + \beta_1*X_{1,i,t} + u_{i,t} \]
                  alpha is error 
                  \[ = \beta_0 + \beta_1*X_{1,i,t} + \alpha_i +  u_{i,t} \]
                  
                  two error components alpha_i, u_it
                  
                  Ignore error structure: OLS \rightarrow unbiased
                                              \rightarrow inefficient
                  
                  \[ \alpha_i ~ N(0, \sigma_\alpha^2)  with u_{it} ~ N(0, \sigma_u^2)\]
                  
                  Estimating: Feasible Generalised Least Squares FGLS
                  
                 \[ E(Cov[X, u]) = 0 \]
                 \[ E(Cov[X, \alpha]) = 0 \] \leftarrow in many contexts this is a critical assumption
                 It is often questionable that individual effects and regressors are uncorrelated.
                 This is NOT required in a fixed-effects model. 
                 
                 
                 \Rightarrow Wald test:
                 
                 \[ (\beta_{FE} - \beta_{RE})(\hat VCOV_{FE} - \hat VCOV_{RE})^(-1)*(\beta_{FE} - \beta_{RE}) \] 
               
                $\Rightarrow$ Hausmann Test:
                Alternative: Variable addition
                
                FE by within  
                
                
                2) plus all X bar i $\Rightarrow$ should be not having any expl power if $E(cov(x, \alpha))$ = 0
                
                Test by F-test whether all $\[ \bar X_{i, s} \]$ have zero parameters or not. 
                
                "Mundlak correction"
                

                <<>>=
                
                RE.NL <- plm(formula.NL, data=dataNL, model = "random")
                cbind(coef(WI.NL), coef(Pool.NL)[2:7])
                
                cbind(coef(WI.NL), coef(Pool.NL)[2:7], coef(RE.NL)[2:7])
                
                summary(RE.NL)
                
                
                phtest(formula.NL, data = dataNL)
                
                
                qchisq(0.95, 6)
                
                mP <- diag(141) %x% crossprod(t(rep(1,4)), rep(1,4))
                @
                
                We got a lower R²
                
                
                Conclusion by chi² test: RE model is inconsistent, we should be using the fixed effects model. 
                Time specific means as result of cronica product are added to the RE model. 
                
                
                
                
                
\end{document}
